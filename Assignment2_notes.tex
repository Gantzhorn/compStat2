\documentclass[a4paper, 11 pt]{article}
\usepackage{datetime}
\usepackage{lipsum}
\setlength{\columnsep}{25 pt}
\usepackage{tabularx,booktabs,caption}
\usepackage{multicol,tabularx,capt-of}
\usepackage{multirow}
\usepackage{subcaption}
\usepackage[utf8]{inputenc}
\usepackage{hyperref}
\usepackage{blkarray}
\usepackage[margin=1.2in]{geometry}
\usepackage{amsthm, amsmath, amssymb, amsfonts, commath, amsthm}
\usepackage{placeins}
\usepackage{graphicx}
\usepackage{mathrsfs}
\usepackage{listings}
\usepackage{lastpage}
\usepackage{enumerate}
\usepackage{subcaption}
\usepackage[danish]{babel}
%\renewcommand{\qedsymbol}{\textit{Q.E.D}}
\newtheorem{theorem}{Theorem}[section]
\addto\captionsdanish{\renewcommand\proofname{Proof}}
\newtheorem{lemma}{Lemma}[section]
\newtheorem{corollary}{corollary}[section]
\usepackage{fancyhdr}
\usepackage{titlesec}
\newcommand{\code}[1]{\texttt{#1}}
\lhead{Anders Gantzhorn - tzk942}
\chead{}
\rhead{09-11-2022}
\pagestyle{fancy}
\setlength{\headheight}{15pt}
\cfoot{\thepage\ af \pageref{LastPage}}
\setcounter{section}{0}
\title{Importance sampling}
\author{Anders G. Kristensen}
\date{09-11-2022}
\begin{document}
\maketitle
\section{Monte Carlo integration}
\subsection{Direct implementation}
\noindent Let $X_1,X_2,\dots$ be i.i.d. uniform variables on $(-1.9, 2)$ and define
\[
    S_n = 30 + \sum_{k = 1}^n X_k
\]
Consider the ruin probability
\[
    p(n) = P(\exists k\leq n : S_k \leq 0)
\]
We use monte carlo integration to estimate $p(100)$. The key insight is that
\[
    p(n) = \mathbb{E}\left[I_{\left(\exists k\leq n : S_k\leq 0\right)}\right]    
\]
As a result we have by the strong law of large numbers
\[
    p(100) \approx \frac{1}{N}\sum_{i = 1}^N I_{\left(\left(\exists k\leq 100 : S_k\leq 0\right),i\right)}
\]
Where $N$ is the number of realizations of the experiment.
\subsection{Importance sampling}
Define:
\[
\varphi(\theta) = \int_{-1.9}^2 \exp\left(\theta z dz\right) = \frac{\exp\left(2\theta\right)-\exp\left(-1.9\theta\right)}{\theta}    
\]
Now let 
\[
    g_{\theta, n}(x) = \frac{1}{\varphi(\theta)^n}\exp{\left(\theta \sum_{k = 1}^n x_k\right)}  
\]
for $x\in(-1.9,2)^n$. Now we easily see that 
\[
    g_{\theta, n} = \prod_{k = 1}^n\left(g_{\theta, 1}\right)_k
\]
With $k = 1,\dots n$ different realizations of $g_{\theta,1}$. So we can manage be sampling from $g_{\theta,1}$. This we do be the quantile method. Firstly, the distribution function is
\[
    G_{\theta,1}(x) = \frac{1}{\varphi(\theta)}\int_{-1.9}^x \exp\left(\theta t\right)dt = \frac{1}{\varphi(\theta)\theta}\left[\exp(\theta t)\right]_{-1.9}^x = \frac{\exp(\theta x)-\exp(-1.9\theta)}{\exp(2\theta)-\exp(-1.9\theta)}
\]
By solving the following equation we get the quantile function
\begin{align*}
    u &= \frac{\exp\left(\theta x\right)-\exp(-1.9\theta)}{\exp(2\theta)-\exp(-1.9\theta)} \\
    Q_{\theta,1}(u) &= \frac{\log\left(u\left(\exp(2\theta)-\exp(-1.9\theta)\right)+\exp(-1.9\theta)\right)}{\theta}
\end{align*}
To simulate from $g_{\theta,1}$ we simulate from the standard uniform distribution and then transform the result with $Q_{\theta,1}(u)$. Analogously to earlier we check whether the event we are taking indicator occurs with our modified $S_n$: $S_{n,IS}$ Now define 
\[
    w^*(X_i) = q^*(X_i)/g_{\theta,1}^*(X_i)    
\]
Where $q^*, g_{\theta,1}^*$ are the unnormalized versions of the uniform density and the distribution from $g_{\theta,1}$ respectively. Thus our weights are
\[
    w(X_i) = \exp\left(-\theta X_i\right)    
\]
Let
\[
w(X_i) = \frac{w^*(X_i)}{\sum_{i = 1}^n w^*(X_i)}  
\]
From the lecture notes we now have 
\[
p(100)_{IS} = \sum_{i = 1}^N w(X_i)\cdot I_{\left(\exists k\leq n :S_{k,IS}\leq 0\right)}
\]
\subsection{Asymptotic results}
\subsubsection{Direct implementation}
By the central limit theorem we know that it for the original estimator holds that
\[
    \hat{p}(100) \overset{\text{approx.}}{\sim} \mathcal{N}\left(p(100), \frac{\sigma_{MC}^2}{N}\right)    
\]
We use $\hat{\sigma}_{MC}^2$ the empirical standard deviation. Thus a $95\%$ confidence-band for $\hat{p}(100)$ is:
\[
    \hat{p}(100) \pm \frac{\hat{\sigma}_{MC}}{\sqrt{n}} 
\]
Bearing in mind that this is an asymptotic result.
\subsubsection{Importance sampling}
From the notes we know that by the $\Delta$-method the asymptotic distribution of $p(100)_{IS}$ is
\[
    \hat{p}(100)_{IS} \overset{\text{approx.}}{\sim} \mathcal{N}\left(p(100), c^{-2}\frac{\sigma_{IS}^2 + p(100)^2 \sigma_w^2-2p(100)\gamma}{N}\right)    
\]
Where $c$ is the normalization constant for our densities, we can estimate this as
\[
    \hat{c} = \frac{1}{N}\sum_{i = 1}^N w^*(X_i)    
\]
Likewise, we have empirical estimates
\begin{align*}
    \hat{\sigma}_{IS}^2 &= V(I_{\left(\exists k\leq n :S_{k,IS}\leq 0\right)}w^*(X_i))\\
    \hat{\gamma} &= cov(I_{\left(\exists k\leq n :S_{k,IS}\leq 0\right)}w^*(X_i), w^*(X_i))\\
    \hat{\sigma}_{w}^2 &= V(w^*(X_i))
\end{align*}
\end{document}