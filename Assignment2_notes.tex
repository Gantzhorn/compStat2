\documentclass[a4paper, 11 pt]{article}
\usepackage{datetime}
\usepackage{lipsum}
\setlength{\columnsep}{25 pt}
\usepackage{tabularx,booktabs,caption}
\usepackage{multicol,tabularx,capt-of}
\usepackage{multirow}
\usepackage{subcaption}
\usepackage[utf8]{inputenc}
\usepackage{hyperref}
\usepackage{blkarray}
\usepackage[margin=1.2in]{geometry}
\usepackage{amsthm, amsmath, amssymb, amsfonts, commath, amsthm}
\usepackage{placeins}
\usepackage{graphicx}
\usepackage{mathrsfs}
\usepackage{listings}
\usepackage{lastpage}
\usepackage{enumerate}
\usepackage{subcaption}
\usepackage[danish]{babel}
%\renewcommand{\qedsymbol}{\textit{Q.E.D}}
\newtheorem{theorem}{Theorem}[section]
\addto\captionsdanish{\renewcommand\proofname{Proof}}
\newtheorem{lemma}{Lemma}[section]
\newtheorem{corollary}{corollary}[section]
\usepackage{fancyhdr}
\usepackage{titlesec}
\newcommand{\code}[1]{\texttt{#1}}
\lhead{Anders Gantzhorn - tzk942}
\chead{}
\rhead{09-11-2022}
\pagestyle{fancy}
\setlength{\headheight}{15pt}
\cfoot{\thepage\ af \pageref{LastPage}}
\setcounter{section}{0}
\title{Importance sampling}
\author{Anders G. Kristensen}
\date{09-11-2022}
\begin{document}
\maketitle
\section{Importance sampling}
\noindent Let $X_1,X_2,\dots$ be i.i.d. uniform variables on $(-1.9, 2)$ and define
\[
    S_n = 30 + \sum_{k = 1}^n X_k
\]
Consider the ruin probability
\[
    p(n) = P(\exists k\leq n : S_k \leq 0)
\]
We use monte carlo integration to estimate $p(100)$. The key insight is that
\[
    p(n) = \mathbb{E}\left[I_{\left(\exists k\leq n : S_k\leq 0\right)}\right]    
\]
As a result we have by the strong law of large numbers
\[
    p(100) \approx \frac{1}{N}\sum_{i = 1}^N I_{\left(\left(\exists k\leq 100 : S_k\leq 0\right),i\right)}
\]
Where $N$ is the number of realizations of the experiment.\\
Define:
\[
\varphi(\theta) = \int_{-1.9}^2 \exp\left(\theta z dz\right) = \frac{\exp\left(2\theta\right)-\exp\left(-1.9\theta\right)}{\theta}    
\]
Now let 
\[
    g_{\theta, n}(x) = \frac{1}{\varphi(\theta)^n}\exp{\left(\theta \sum_{k = 1}^n x_k\right)}  
\]
for $x\in(-1.9,2)^n$. Now we easily see that 
\[
    g_{\theta, n} = \prod_{i = 1}^n\left(g_{\theta, 1}\right)_i
\]
With $i = 1,\dots n$ different realizations of $g_{\theta,1}$. So we can manage be sampling from $g_{\theta,1}$. This we do be the quantile method. Firstly, the distribution function is
\[
    G_{\theta,1}(x) = \frac{1}{\varphi(\theta)}\int_{-1.9}^x \exp\left(\theta t\right)dt = \frac{1}{\varphi(\theta)\theta}\left[\exp(\theta t)\right]_{-1.9}^x = \frac{\exp(\theta x)-\exp(-1.9\theta)}{\exp(2\theta)-\exp(-1.9\theta)}
\]
By solving the following equation we get the quantile function
\begin{align*}
    u &= \frac{\exp\left(\theta x\right)-\exp(-1.9\theta)}{\exp(2\theta)-\exp(-1.9\theta)} \\
    Q_{\theta,1}(u) &= \frac{\log\left(u\left(\exp(2\theta)-\exp(-1.9\theta)\right)+\exp(-1.9\theta)\right)}{\theta}
\end{align*}
\end{document}